\section{Introduction}
A Binary Search Tree (BST) implements the dictionary abstract data type. In a BST, all the values in a node's left subtree are less than the node value and all the values in the node's right subtree are greater than the node value. Duplicates are not allowed. A BST supports three main operations, viz.: search, insert and delete. Search($x$) determines if $x$ is present in the tree. Insert($x$) adds key $x$ to the tree if it is not already present. Delete($x$) removes the key $x$ from the tree if it is present. \par
Several algorithms have been proposed for non-blocking binary search trees. Ellen \textit{et al.} proposed the first practical lock-free algorithm for a concurrent binary search tree in~\cite{EllFat+:2010:PODC}. Their algorithm uses an external (or leaf-oriented) search tree in which only the leaf nodes store the actual keys; keys stored at internal nodes are used for routing purposes only. Howley and Jones have proposed another lock-free algorithm for a concurrent binary search tree in~\cite{HowJon:2012:SPAA} which uses an internal search tree in which both the leaf nodes as well as the internal nodes store the actual keys. One advantage of using an internal search tree is that its memory footprint is half that of an external search tree. For large size trees, search time dominates and our experimental results shows that internal BST tend to perform better than external BSTs. Natarajan and Mittal have proposed another lock-free external BST in~\cite{Natarajan:2014:PPOPP}. The key change in this algorithm is that it operates at edge level and the ones described by Ellen \textit{et al.}~\cite{EllFat+:2010:PODC} and Howley and Jones~\cite{HowJon:2012:SPAA}  operates on node level. A node (or edge) level operation mean that nodes (or edges) are marked for deletion. Operating at edge level blocks fewer operations than operating at node level. So edge level operation provides more concurrency when there is high contention. Hence the algorithm described by Natarajan and Mittal~\cite{Natarajan:2014:PPOPP} performs well for smaller trees with high contention. We have designed our algorithm to get the benefits of both the worlds. Our algorithm like the one proposed by Howley and Jones~\cite{HowJon:2012:SPAA} is an internal BST and like the one proposed by Natarajan and Mittal~\cite{Natarajan:2014:PPOPP} operate on edge level.