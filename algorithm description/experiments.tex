\section{Experimental Evaluation}
\subsubsection{Experimental Setup:} We conducted our experiments on an X86\_64 AMD Opteron 6276 machine running GNU/Linux operating system. We used gcc 4.6.3 compiler with optimization flag set to $O3$. The Table I shows the hardware features of this machine. All implementations were written in C++. To compare the performance of different implementations, we considered the following parameters:
\begin{enumerate}
\item \textbf{Maximum Tree Size:} This depends on the size of the key space. We consider five different key ranges: 1000(1K), 10,000 (10K), 100,000 (100K), 1 million (1M) and 10 million (10M) keys. To capture only the steady state behaviour we \textit{pre-populated} the tree to 50$\%$ of its maximum size, prior to starting the simulation run.
\item \textbf{Relative Distribution of Various Operations:} We consider three different workload  distributions: (a) \textit{read-dominated} workload : 90$\%$ search, 9$\%$ insert and 1$\%$ delete, (b) \textit{mixed} workload : 70$\%$ search, 20$\%$ insert and 10$\%$ delete and (c) \textit{write-dominated} workload : 0$\%$ search, 50$\%$ insert and 50$\%$ delete
\item \textbf{Maximum Degree of Concurrency:} This depends on number of threads concurrently operating on the tree. We varied the number of threads from 1 to 128 in increments in powers of 2.
\end{enumerate}
\begin{table}[ht]
\small
\caption{hardware features}
\begin{tabular}{|l|l|}
\hline
CPU sockets      & 4         \\ \hline
Cores per socket & 8         \\ \hline
Threads per core  & 2         \\ \hline
Clock frequency   & 2.3 GHz   \\ \hline
L1 cache (I/D)    & 64KB/16KB \\ \hline
L2 cache          & 2 MB      \\ \hline
L3 cache          & 6 MB      \\ \hline
Memory            & 256 GB    \\ \hline
\end{tabular}
\quad
\begin{tabular}{|c|c|c|c|c|}
\hline
\multirow{2}{*}{\textbf{Algorithm}} & \multicolumn{2}{c|}{\textbf{\begin{tabular}[c]{@{}c@{}}\# of \\ Objects\\  Allocated\end{tabular}}} & \multicolumn{2}{c|}{\textbf{\begin{tabular}[c]{@{}c@{}}\#of Atomic \\ Instructions\\  Executed\end{tabular}}} \\ \cline{2-5} 
                                    & \textbf{Ins}                                     & \textbf{Del}                                     & \textbf{Ins}                                          & \textbf{Del}                                          \\ \hline
\textbf{Ellen \& \textit{et al}.}                      & 4                                                & 2                                                & 3                                                     & 4                                                     \\ \hline
\textbf{Howley \& Jones}                     & 2                                                & 1                                                & 3                                                     & upto 9                                                \\ \hline
\textbf{Natarajan \& Mittal}        & 2                                                & 0                                                & 1                                                     & 3                                                     \\ \hline
\textbf{This work}                  & 1                                                & 1                                                & 1                                                     & upto 6                                                \\ \hline
\end{tabular}
\end{table}
We compared the performance of different implementations with respect to two metrics:
\begin{enumerate}
	\item \textbf{System Throughput:} it is defined as the number of operations executed per unit time.
	\item \textbf{Avg Seek Length:} it is defined as the average length of the \accesspath of a seek operation.
\end{enumerate}
\newpage
\begin{tikzpicture}
\footnotesize
%\scriptsize
	\begin{groupplot}[group style={group size= 2 by 4},height=5cm,width=6.08cm]
		\nextgroupplot[title=Read-Dominated,ylabel={\small keyrange: \small 1K},symbolic x coords={1,2,4,8,16,32,64,128},xtick=data]
				\addplot[red,mark=triangle] 	table{Data/HJ-BST.dat}; \label{plots:HJ-BST}
				\addplot[blue,mark=asterisk] 	table{Data/NM-BST.dat}; \label{plots:NM-BST}
				\addplot[green,mark=square] 	table{Data/RM-BST.dat}; \label{plots:RM-BST}
				\coordinate (top) at (rel axis cs:0,1);% coordinate at top of the first plot
		\nextgroupplot[title=Mixed,symbolic x coords={1,2,4,8,16,32,64,128},xtick=data]
				\addplot[red,mark=triangle] 	table{Data/HJ-BST.dat};
				\addplot[blue,mark=asterisk] 	table{Data/NM-BST.dat};
				\addplot[green,mark=square] 	table{Data/RM-BST.dat};
		\nextgroupplot[ylabel={\small keyrange: \small 10K},symbolic x coords={1,2,4,8,16,32,64,128},xtick=data]
				\addplot[red,mark=triangle] 	table{Data/HJ-BST.dat};
				\addplot[blue,mark=asterisk] 	table{Data/NM-BST.dat};
				\addplot[green,mark=square] 	table{Data/RM-BST.dat};
		\nextgroupplot[symbolic x coords={1,2,4,8,16,32,64,128},xtick=data]
				\addplot[red,mark=triangle] 	table{Data/HJ-BST.dat};
				\addplot[blue,mark=asterisk] 	table{Data/NM-BST.dat};
				\addplot[green,mark=square] 	table{Data/RM-BST.dat};
		\nextgroupplot[ylabel={\small keyrange: \small 100K},symbolic x coords={1,2,4,8,16,32,64,128},xtick=data]
				\addplot[red,mark=triangle] 	table{Data/HJ-BST.dat};
				\addplot[blue,mark=asterisk] 	table{Data/NM-BST.dat};
				\addplot[green,mark=square] 	table{Data/RM-BST.dat};
		\nextgroupplot[symbolic x coords={1,2,4,8,16,32,64,128},xtick=data]
				\addplot[red,mark=triangle] 	table{Data/HJ-BST.dat};
				\addplot[blue,mark=asterisk] 	table{Data/NM-BST.dat};
				\addplot[green,mark=square] 	table{Data/RM-BST.dat};
		\nextgroupplot[xlabel={Number of Threads},ylabel={\small keyrange: \small 1M},symbolic x coords={1,2,4,8,16,32,64,128},xtick=data]
				\addplot[red,mark=triangle] 	table{Data/HJ-BST.dat};
				\addplot[blue,mark=asterisk] 	table{Data/NM-BST.dat};
				\addplot[green,mark=square] 	table{Data/RM-BST.dat};
		\nextgroupplot[xlabel={Number of Threads},symbolic x coords={1,2,4,8,16,32,64,128},xtick=data]
				\addplot[red,mark=triangle] 	table{Data/HJ-BST.dat};
				\addplot[blue,mark=asterisk] 	table{Data/NM-BST.dat};
				\addplot[green,mark=square] 	table{Data/RM-BST.dat};
				\coordinate (bot) at (rel axis cs:0,0);% coordinate at bottom of the last plot
	\end{groupplot}
	%\path (top-|current bounding box.west)-- node[anchor=south,rotate=90] {\small Throughput (million ops/sec)} (bot-|current bounding box.west);
	\node[right,rotate=90] at (-1.4,-7.2){\small ~Throughput (million ops/sec)};
	\path (top|-current bounding box.north)-- coordinate(legendpos) (bot|-current bounding box.north);
	\matrix[matrix of nodes, anchor=south, draw, inner sep=0.2em, draw] at ([yshift=1ex]legendpos)
  {
    \ref{plots:HJ-BST}& HJ-BST&[5pt]
    \ref{plots:NM-BST}& NM-BST&[5pt]
    \ref{plots:RM-BST}& RM-BST\\
	};
\end{tikzpicture}
\newpage
\begin{tikzpicture}
	\begin{groupplot}[group style={group size= 1 by 4},height=5cm,width=6.08cm]
		\nextgroupplot[title=Write-Dominated,ylabel={\small keyrange: 1K},symbolic x coords={1,2,4,8,16,32,64,128},xtick=data]
				\addplot[red,mark=triangle] 	table{Data/HJ-BST.dat}; 
				\addplot[blue,mark=asterisk] 	table{Data/NM-BST.dat}; 
				\addplot[green,mark=square] 	table{Data/RM-BST.dat}; 
				\coordinate (top) at (rel axis cs:0,1);% coordinate at top of the first plot
		\nextgroupplot[ylabel={\small keyrange: 10K},symbolic x coords={1,2,4,8,16,32,64,128},xtick=data]
				\addplot[red,mark=triangle] 	table{Data/HJ-BST.dat};
				\addplot[blue,mark=asterisk] 	table{Data/NM-BST.dat};
				\addplot[green,mark=square] 	table{Data/RM-BST.dat};
		\nextgroupplot[ylabel={\small keyrange: 100K},symbolic x coords={1,2,4,8,16,32,64,128},xtick=data]
				\addplot[red,mark=triangle] 	table{Data/HJ-BST.dat};
				\addplot[blue,mark=asterisk] 	table{Data/NM-BST.dat};
				\addplot[green,mark=square] 	table{Data/RM-BST.dat};
		\nextgroupplot[xlabel={Number of Threads},ylabel={\small keyrange: 1M},symbolic x coords={1,2,4,8,16,32,64,128},xtick=data]
				\addplot[red,mark=triangle] 	table{Data/HJ-BST.dat};
				\addplot[blue,mark=asterisk] 	table{Data/NM-BST.dat};
				\addplot[green,mark=square] 	table{Data/RM-BST.dat};
				\coordinate (bot) at (rel axis cs:0,0);% coordinate at bottom of the last plot
	\end{groupplot}
	%\path (top-|current bounding box.west)-- node[anchor=south,rotate=90] {Throughput (million ops/sec)} (bot-|current bounding box.west);
	\node[right,rotate=90] at (-1.4,-7.2){\small ~Throughput (million ops/sec)};
	\path (top|-current bounding box.north)-- coordinate(legendpos) (bot|-current bounding box.north);
	\matrix[matrix of nodes, anchor=south, draw, inner sep=0.2em, draw] at ([yshift=1ex]legendpos)
  {
    \ref{plots:HJ-BST}& HJ-BST&[5pt]
    \ref{plots:NM-BST}& NM-BST&[5pt]
    \ref{plots:RM-BST}& RM-BST\\
	};
\end{tikzpicture}